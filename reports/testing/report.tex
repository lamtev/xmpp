\documentclass[a4paper,14pt]{extarticle}

\usepackage[utf8x]{inputenc}
\usepackage[T1]{fontenc}
\usepackage[russian]{babel}
\usepackage{hyperref}
\usepackage{indentfirst}
\usepackage{here}
\usepackage{array}
\usepackage{graphicx}
\usepackage{grffile}
\usepackage{caption}
\usepackage{subcaption}
\usepackage{chngcntr}
\usepackage{amsmath}
\usepackage{amssymb}
\usepackage{pgfplots}
\usepackage{pgfplotstable}
\usepackage[left=2cm,right=2cm,top=2cm,bottom=2cm,bindingoffset=0cm]{geometry}
\usepackage{multicol}
\usepackage{multirow}
\usepackage{titlesec}
\usepackage{listings}
\usepackage{color}
\usepackage{longtable}
\usepackage{enumitem}
\usepackage{cmap}
\usepackage{tikz}

\usetikzlibrary{shapes,arrows}

\definecolor{green}{rgb}{0,0.6,0}
\definecolor{gray}{rgb}{0.5,0.5,0.5}
\definecolor{purple}{rgb}{0.58,0,0.82}

\lstset{
	language={Java},
	inputpath={../../},
	backgroundcolor=\color{white},
	commentstyle=\color{green},
	keywordstyle=\color{blue},
	numberstyle=\scriptsize\color{gray},
	stringstyle=\color{purple},
	basicstyle=\tiny,
	breakatwhitespace=false,
	breaklines=true,
	captionpos=b,
	keepspaces=true,
	numbers=left,
	numbersep=5pt,
	showspaces=false,
	showstringspaces=false,
	showtabs=false,
	tabsize=8,
	frame=single,
}

\renewcommand{\le}{\ensuremath{\leqslant}}
\renewcommand{\leq}{\ensuremath{\leqslant}}
\renewcommand{\ge}{\ensuremath{\geqslant}}
\renewcommand{\geq}{\ensuremath{\geqslant}}
\renewcommand{\epsilon}{\ensuremath{\varepsilon}}
\renewcommand{\phi}{\ensuremath{\varphi}}
\renewcommand{\thefigure}{\arabic{figure}}
\def\code#1{\texttt{#1}}

\titleformat*{\section}{\large\bfseries} 
\titleformat*{\subsection}{\normalsize\bfseries} 
\titleformat*{\subsubsection}{\normalsize\bfseries} 
\titleformat*{\paragraph}{\normalsize\bfseries} 
\titleformat*{\subparagraph}{\normalsize\bfseries} 

\counterwithin{figure}{section}
\counterwithin{equation}{section}
\counterwithin{table}{section}
\newcommand{\sign}[1][5cm]{\makebox[#1]{\hrulefill}}
\newcommand{\equipollence}{\quad\Leftrightarrow\quad}
\newcommand{\no}[1]{\overline{#1}}
\graphicspath{{figs/}}
\captionsetup{justification=centering,margin=1cm}
\def\arraystretch{1.3}
\setlength\parindent{5ex}
\titlelabel{\thetitle.\quad}

\setitemize{itemsep=0em}
\setenumerate{itemsep=0em}

\tikzstyle{startstop} = [
	rectangle,
	align=center,
	rounded corners,
	text width=10em,
	text centered,
	draw=black
]
\tikzstyle{process} = [
	rectangle,
	align=center,
	text width=20em,
	text centered,
	draw=black
]
\tikzstyle{decision} = [
	diamond,
	aspect=4,
	align=center,
	inner sep=0pt,
	text width=10em,
	text centered,
	node distance=5em,
	draw=black
]
\tikzstyle{line} = [
	draw=black,
	thick,
	->,
	>=stealth,
	-latex'
]

\begin{document}

\begin{titlepage}
\begin{center}
	САНКТ-ПЕТЕРБУРГСКИЙ ПОЛИТЕХНИЧЕСКИЙ УНИВЕРСИТЕТ\\ ПЕТРА ВЕЛИКОГО\\[0.3cm]
	\par\noindent\rule{10cm}{0.4pt}\\[0.3cm]
	Институт компьютерных наук и технологий \\[0.3cm]
	Кафедра компьютерных систем и программных технологий\\[4cm]
	
	Отчет по лабораторной работе\\[3mm]
	Дисциплина: <<Тестирование программного обеспечения>>\\[3mm]
	Тема: <<Тестирование реализации протокола XMPP>>\\[7cm]
\end{center}

\begin{flushleft}
	\hspace*{5mm} Выполнил студент гр. 43501/3  \hspace*{2.5cm}\sign[3cm]\hfill А.Ю. Ламтев\\
	\hspace*{10.4cm} (подпись)\\[3mm]
	\hspace*{5mm} Преподаватель \hspace*{6.0cm}\sign[3cm]\hfill М.Х. Ахин\\
	\hspace*{10.4cm} (подпись)\\[3mm]
	\hspace*{11.1cm} <<\sign[7mm]>> \sign[27mm] \the\year\hspace{1mm} г.
\end{flushleft}

\vfill

\begin{center}
	Санкт-Петербург\\
	\the\year
\end{center}
\end{titlepage}
\addtocounter{page}{1}

\tableofcontents
\newpage

\section{Введение}

В качестве объекта тестирования была выбрана реализация протокола \code{XMPP} на языке Java. Реализация протокола \code{XMPP} разрабатывалась частично согласно методологии \code{Test Driven Development} (TDD). Некоторые автоматические модульные тесты были написаны еще до реализации самих модулей. А некоторые тесты были написаны уже пост фактум, что противоречит методологии TDD. Метрикой оценивания качества тестирования был выбран процент покрытия. Для выполнения автоматических тестов была автоматизирована непрерывная интеграция: при пуше изменений в гит репозиторий на сервере непрерывной интеграции выполняется все тесты и результат их выполнения отправляется на почту.

\section{Автоматические модульные тесты}

Для модульного автоматического тестирования использовалась библиотека \code{JUnit Jupiter} 5.4.1.

Для классов, содержащих какую-либо бизнес-логику, были написаны тестовые классы. Рассмотрим тестируемые классы и тестовые сценарии:

\begin{itemize}
	\item \textbf{Тестируемый класс:} \code{XmppStreamParser}. Реализует функциональность парсинга XMPP модулей (различных XML элементов) среди непрерывного потока текстовых данных.
	
	\textbf{Тестирующий класс:} \code{XmppStreamParserTest}
	
	\textbf{Тестовые сценарии:} на вход подаются данные, содержащие различные XMPP модули, такие, как \textbf{stream header}, \textbf{stream features}, \textbf{SASL auth}, \textbf{stanza}, \textbf{error} и другие. Выходные объекты, соответствующие поданным на вход XMPP модулям, сравниваются с ожидаемыми.
	
	
	\item \textbf{Тестируемый класс:} \code{XmppStreamParserStrategyCache}. Реализует функциональность кэширования и дедупликации объектов стратегий.
	
	\textbf{Тестирующий класс:} \code{XmppStreamParserStrategyCacheTest}
	
	\textbf{Тестовые сценарии:} проверка того, что кэш отдает не нули. И проверка того, что кэш для одинаковых ключей отдает одни и те же объекты (сравнение не по значению, а по ссылке).
	
	
	\item \textbf{Тестируемый класс:} \code{XmppUnitSerializer}. Реализует функциональность сериализатора XMPP модулей в массив байтов.
	
	\textbf{Тестирующий класс:} \code{XmppUnitSerializerTest}
	
	\textbf{Тестовые сценарии:} на вход подаются все виды XMPP модулей. Выходные массивы байтов сравниваются с ожидаемыми.
	
	
	\item \textbf{Тестируемый класс:} \code{XmppInputStream}. Реализует поток, который позволяет читать из сокета сразу XMPP модули. В терминологии протокола данный класс может выполнять роль Initial Stream для сервера и Response stream для клиента.
	
	\textbf{Тестирующий класс:} \code{XmppInputStreamTest}
	
	\textbf{Тестовые сценарии:} на вход подается поток текстовых данных, содержащих, XMPP модули. Выходные данные сравниваются с ожидаемыми.
	
	
	\item  \textbf{Тестируемый класс:} \code{XmppOutputStream}. Реализует поток, который позволяет писать в сокет XMPP модули. В терминологии протокола данный класс может выполнять роль Initial Stream для клиента и Response stream для сервера.
	
	\textbf{Тестирующий класс:} \code{XmppOutputStreamTest}
	
	\textbf{Тестовые сценарии:} на вход подаются XMPP модули. Выходные данные сравниваются с ожидаемыми.
\end{itemize}

\section{Процент покрытия}

Процент покрытия считается с помощью утилиты \code{JaCoCo}. Ее работа организована с помощью \code{Gradle} плагина утилиты. Для детализации покрытия используется сервис \href{https://codecov.io/gh/lamtev/xmpp}{\code{codecov.io}}\footnote{\href{https://codecov.io/gh/lamtev/xmpp}{https://codecov.io/gh/lamtev/xmpp}}.

\textbf{Процент покрытия:} 61.

\section{Непрерывная интеграция}

Сервер непрерывной интеграции развернут с помощью сервиса \href{https://travis-ci.org/lamtev/xmpp}{\code{travis-ci.org}}\footnote{\href{https://travis-ci.org/lamtev/xmpp}{https://travis-ci.org/lamtev/xmpp}}.

Конфиг \code{travis-ci} представлен в листинге \ref{lst:travis.yml} \lstinputlisting[caption={\code{.travis.yml}},label={lst:travis.yml},basicstyle=\scriptsize]{../.travis.yml}

Сборка проекта и прогон тестов осуществляется внутри \href{https://cloud.docker.com/u/lamtev/repository/docker/lamtev/java}{\code{docker контейнера}}\footnote{\href{https://cloud.docker.com/u/lamtev/repository/docker/lamtev/java}{https://cloud.docker.com/u/lamtev/repository/docker/lamtev/java}}. При успешной сборке результаты расчета процента покрытия отправляются в сервис \code{codecov.io}, о котором упомянуто в предыдущем разделе.


\section{Заключение}

Разработанные тесты позволяли своевременно на этапе разработке обнаруживать проблемы. Ипользуемые инструменты тестирования позволили повысить качество реализации протокола XMPP. Проект с тестами расположен на \href{https://github.com/lamtev/xmpp}{\code{https://github.com/lamtev/xmpp}}.

\end{document}
